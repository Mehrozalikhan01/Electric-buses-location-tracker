\documentclass[12pt]{article}
\usepackage[utf8]{inputenc}
\usepackage[a4paper, margin=1in]{geometry}
\usepackage{times}
\usepackage{enumitem}
\usepackage{setspace}

\title{\textbf{Electric Bus Tracking System Proposal}}
\author{}
\date{}

\begin{document}

\maketitle
\onehalfspacing

\section*{Introduction}

The increasing focus on transportation by the Government of Punjab has led to the introduction of electric buses in the Mianwali district. These buses provide essential transportation facilities to the citizens of Mianwali. However, a major challenge faced by the urban area is the absence of a system for monitoring transportation, especially in Mianwali district.  

The Electric Bus Tracking System addresses this challenge by providing a platform to monitor the real-time location of these buses. This system integrates a GPS module for tracking and will provide accurate details about each bus. The system will include user interfaces where passengers can check details such as bus locations, routes, and estimated arrival times. It will facilitate both passengers and authorities by managing the real-time movement of buses. Through the implementation of this system, the transportation department will enhance service reliability, reduce waiting times, and ensure better control over electric bus operations.

\section*{Problem Statement}

The current electric bus system is facing challenges in providing reliable and real-time information to the public of Mianwali. The transport administration experiences difficulties in determining the best routes, monitoring live locations, and handling inconveniences that may occur during operations.  

The absence of such a system in Mianwali affects operational decision-making and the efficient management of energy consumption. Moreover, passengers cannot track the live location or arrival time of buses. Therefore, there is a strong need for a real-time Electric Bus Tracking System that can display accurate live locations and estimated arrival times. Implementing this system will help passengers in the city track buses more effectively.

\section*{Objectives}

The main objective of this project is to design and document a complete software-based solution that enables real-time tracking and management of electric buses in Mianwali.

The specific objectives of the system are as follows:

\begin{itemize}[leftmargin=1.2cm]
    \item To identify system requirements through consultation with stakeholders in Mianwali District.
    \item To develop a system that uses GPS for real-time bus tracking.
    \item To design a user-friendly interface for passengers to view live bus locations, estimated arrival times, and route information.
    \item To enable transport administrators to monitor bus operations and ensure timely responses to operational issues.
    \item To contribute to public transport in Mianwali by enhancing service reliability, reducing passenger waiting time, and promoting the use of electric buses.
\end{itemize}

\end{document}
